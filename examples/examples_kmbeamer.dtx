% \iffalse 
%<*internal>
\input docstrip.tex
\keepsilent
\usedir{tex/latex/kmbeamer/examples}
\let\MetaPrefix\relax
\preamble
Copyright (c) 2011 Kazuki Maeda <kmaeda@users.sourceforge.jp>

Distributable under the MIT License:
http://www.opensource.org/licenses/mit-license.php

\endpreamble
\askforoverwritefalse

\let\MetaPrefix\DoubleperCent
\generate{\file{example_Blackboard.tex}{\from{examples_kmbeamer.dtx}{Blackboard}}}
\endbatchfile
%</internal>

\documentclass{beamer}

%<Blackboard>\usetheme{Blackboard}
\usefonttheme{luatexja}
\setbeamertemplate{theorems}[normal font]

\usepackage{unicode-math}
\setmathfont{XITSMath}
\ltjsetparameter{jacharrange={-3}}

\begin{document}
\title{\texttt{kmbeamer} のテスト}
%<Blackboard>\subtitle{Blackboard 編}
\author{前田一貴\footnote{\texttt{kmaeda@users.sourceforge.jp}}}

\begin{frame}
  \maketitle
\end{frame}

\begin{frame}{目次}
  \tableofcontents
\end{frame}

\section{テスト}

\begin{frame}{テスト1}
  これはテストです.

  \pause

  \begin{enumerate}
  \item リスト1\pause
  \item リスト2\pause
  \item リスト3
  \end{enumerate}

  \pause

  \begin{itemize}
  \item リスト1\pause
  \item リスト2\pause
  \item リスト3
  \end{itemize}
\end{frame}

\begin{frame}{テスト2}
  数式のテスト.

  \begin{theorem}[Gauss 積分]\upshape
    以下の等式が成り立つ:
    \begin{equation}
      \int_{-\infty}^\infty \mathrm{e}^{-x^2}\mathrm{d}x=\sqrt{\pi}.
    \end{equation}
  \end{theorem}
\end{frame}

\section{もっとテスト}
\begin{frame}
あああああああああああああああああああああああああああああああああああああああああああああああああああああああああああああああああああああああああああああああああああああああああああああああああああああああああああああああああああああああああああああああああああああああああああああああああああああああああああああああああああああああああああああああああああああああああああああああああああああああああああああああああああああああああああああああああああああああああああああああああああああああああああああああああああああああああああああああああああああああああああああああああああああああああああああああああああああああああああああああああああああああああああああああああああああああああああああああああああああああああああああああああああああああああああああああああああああああああああああああああああああああああああああああああああああああああああああああああああああああああああああああああああああああああああああああああああああああああああああああああああああああああああああああああああああああああああああああああああああああああああああああ
\end{frame}
\end{document}
